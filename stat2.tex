% Options for packages loaded elsewhere
\PassOptionsToPackage{unicode}{hyperref}
\PassOptionsToPackage{hyphens}{url}
%
\documentclass[
]{article}
\usepackage{amsmath,amssymb}
\usepackage{iftex}
\ifPDFTeX
  \usepackage[T1]{fontenc}
  \usepackage[utf8]{inputenc}
  \usepackage{textcomp} % provide euro and other symbols
\else % if luatex or xetex
  \usepackage{unicode-math} % this also loads fontspec
  \defaultfontfeatures{Scale=MatchLowercase}
  \defaultfontfeatures[\rmfamily]{Ligatures=TeX,Scale=1}
\fi
\usepackage{lmodern}
\ifPDFTeX\else
  % xetex/luatex font selection
\fi
% Use upquote if available, for straight quotes in verbatim environments
\IfFileExists{upquote.sty}{\usepackage{upquote}}{}
\IfFileExists{microtype.sty}{% use microtype if available
  \usepackage[]{microtype}
  \UseMicrotypeSet[protrusion]{basicmath} % disable protrusion for tt fonts
}{}
\makeatletter
\@ifundefined{KOMAClassName}{% if non-KOMA class
  \IfFileExists{parskip.sty}{%
    \usepackage{parskip}
  }{% else
    \setlength{\parindent}{0pt}
    \setlength{\parskip}{6pt plus 2pt minus 1pt}}
}{% if KOMA class
  \KOMAoptions{parskip=half}}
\makeatother
\usepackage{xcolor}
\usepackage[margin=1in]{geometry}
\usepackage{graphicx}
\makeatletter
\def\maxwidth{\ifdim\Gin@nat@width>\linewidth\linewidth\else\Gin@nat@width\fi}
\def\maxheight{\ifdim\Gin@nat@height>\textheight\textheight\else\Gin@nat@height\fi}
\makeatother
% Scale images if necessary, so that they will not overflow the page
% margins by default, and it is still possible to overwrite the defaults
% using explicit options in \includegraphics[width, height, ...]{}
\setkeys{Gin}{width=\maxwidth,height=\maxheight,keepaspectratio}
% Set default figure placement to htbp
\makeatletter
\def\fps@figure{htbp}
\makeatother
\setlength{\emergencystretch}{3em} % prevent overfull lines
\providecommand{\tightlist}{%
  \setlength{\itemsep}{0pt}\setlength{\parskip}{0pt}}
\setcounter{secnumdepth}{-\maxdimen} % remove section numbering
\ifLuaTeX
  \usepackage{selnolig}  % disable illegal ligatures
\fi
\usepackage{bookmark}
\IfFileExists{xurl.sty}{\usepackage{xurl}}{} % add URL line breaks if available
\urlstyle{same}
\hypersetup{
  hidelinks,
  pdfcreator={LaTeX via pandoc}}

\author{}
\date{\vspace{-2.5em}}

\begin{document}

{
\setcounter{tocdepth}{5}
\tableofcontents
}

\subsection{\texorpdfstring{\textbf{Strange Hadron Production in Au-Au
Collisions}}{Strange Hadron Production in Au-Au Collisions}}\label{strange-hadron-production-in-au-au-collisions}

\includegraphics{https://img.shields.io/badge/Using-Python-blue}
\includegraphics{https://img.shields.io/badge/-one\%20way\%20Anova-success}
\includegraphics{https://img.shields.io/badge/-two\%20sample\%20independent\%20t\%20test-success}

\subsubsection{1. Figure}\label{figure}

{[}Fig. The antibaryon-to-baryon ratios as a function of ⟨N\_part⟩ from
Au + Au collisions in different energy levels.{]}

\subsubsection{2. Goal}\label{goal}

To analyze the production of strange hadrons in Au-Au collisions by
modeling the average transverse mass and antibaryon-to-baryon ratios,
with the aim of identifying significant differences in data across
different center-of-mass energies ranging from 7.7 to 39 GeV.

\subsubsection{3. Methodology \& Summary}\label{methodology-summary}

\begin{itemize}
\tightlist
\item
  We used polynomial model fitting to describe the trends in average
  transverse mass and antibaryon-to-baryon ratios for various energy
  levels. For hypothesis testing, we employed a two-sample t-test to
  analyze the antibaryon-to-baryon ratio and a one-way ANOVA test for
  the transverse mass. The results indicated significant differences in
  both variables at different energy levels, with p-values 0.003 and
  0.01, respectively, rejecting the null hypotheses.
\item
  Our analysis revealed that higher energy levels result in an increased
  magnitude of both the antibaryon-to-baryon ratio and transverse mass.
  The best-fit model for the antibaryon-to-baryon ratio had an R-squared
  value of 0.9999 for the 11.5 GeV energy level, while the transverse
  mass model at 19.6 GeV also achieved an R-squared value of 0.9999.
  These findings suggest that there is a strong dependence of the
  produced hadron properties on the center-of-mass energy, contributing
  to our understanding of particle dynamics in heavy ion collisions.
\end{itemize}

\subsubsection{4. Code}\label{code}

Please click \href{files/PHYS_267_Final_Project.pdf}{HERE} for the
analysis report and code.

\end{document}
